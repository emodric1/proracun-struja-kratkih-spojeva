\subsection{Određivanje vrijednosti upravljačkog napona pumpe}

S ciljem određivanja tražene vrijednosti upravljačkog napona pumpe $u_{0}$ koja održava nivo u rezervoaru na nivou $h_{0}$, na ulaz sistema su dovođene različite vrijednosti napona u opsegu 0 - 10 V. Određeno je da se za vrijednost napona $u_{0} = 8.1 V$ na izlazu održava vrijednost nivoa $h_{0} = 17 cm$.

\subsection{Odziv sistema na dovođenje step ulaza}

Kada se umjesto napona $u_{0}$ na ulaz dovede step amplitude 10 V, na izlazu sistema se dobije odziv prikazan na slici \ref{referenca}.

\begin{center}
    \captionsetup{type=figure}
    \begin{center}
        \includegraphics[width=0.8\textwidth]{slike/imeslike.eps}
        \caption{ime slike}
        \label{referenca}
    \end{center}
\end{center}

Kako je dinamika sistema modelirana prenosnom funkcijom integratora, možemo zaključiti da je dobiveni odziv sistema očekivan. 

\subsection{Određivanje pojačanja K iz snimljenog odziva}


Koeficijent pojačanja K, koji odgovara snimljenom odzivu sistema, određujemo sa linearnog dijela karakteristike prikazane na slici \ref{S4}. Koeficijent pojačanja određujemo na način da odaberemo 2 susjedna uzorka, te potom razliku njihovih amplituda podijelimo sa korakom (koeficijent pojačanja je zapravo jednak nagibu linearnog dijela karakteristike). \\

Pojačanje aproksimativnog modela sistema dobivenog na opisani način, uz očitavanje potrebnih podataka sa slike \ref{S4}, iznosi $K = 0.05$.

\subsection{Diskretizacija sistema i predstavljanje sistema u prostoru stanja}

Sistem zadan prenosnom funkcijom 1.2 najprije diskretiziramo korakom diskretizacije $T = 0.01 s$, a potom ga prevodimo u prostor stanja korištenjem Matlab koda prikazanog u nastavku.

\lstinputlisting[language=python,style=Style1]{diskretizacija_PS.m}

Nakon pokretanja koda dobijamo zapis u prostoru stanja:

\begin{equation}
    x[k + 1] = x[k] + 0.03125 \cdot u[k]
\end{equation}

\begin{equation}
    y[k] = 0.016 \cdot x[k]
\end{equation}


\subsection{Određivanje pojačanja LQR}
Napisan je M-fajl koji riješava Riccatijevu jednačinu i određena su pojačanja regulatora za sve
slučajeve iz tabele date u postavci zadatka. Korišteni kod je prikazan u nastavku.
\lstinputlisting[language=python,style=Style1]{rikati.m}

\noindent Za podatke $h_0=17$ i $h_N=22$, dobijena vremenska promjena vektora ulaza na pumpu i stanja sistema je prikazana na slici \ref{grafik1}.

\begin{center}
    \captionsetup{type=figure}
    \begin{center}
        \includegraphics[width=\textwidth]{slike/12_22_120.png}
        \caption{Vremenska promjena ulaza i stanja}
        \label{grafik1}
    \end{center}
\end{center}

Ovakvo rješenje je dobijeno za vrijednost $T_N(s)=120s$, budući da za vrijednosti $T_N(s)=10s$ i $T_N(s)=20s$ sistem ne dostigne stacionarno stanje, kao što se vidi na slikama \ref{grafik2} i \ref{grafik3}.

\begin{figure}[H]
  \centering
  \begin{minipage}[b]{0.47\textwidth}
    \includegraphics[width=\textwidth]{slike/10.png}
    \caption{Grafici za $T_N=10s$}
    \label{grafik2}
  \end{minipage}
  \hfill
  \begin{minipage}[b]{0.47\textwidth}
    \includegraphics[width=\textwidth]{slike/20.png}
    \caption{Grafici za T_N=20s}
    \label{grafik3}
  \end{minipage}
\end{figure}

\subsection{Simulacija ponašanja sistema sa dizajniranim LQR}
U cilju realizacije ovog dijela vježbe, izvršena je modifikacija originalnog
modela sistema, u smislu da se kao ulazni signal u pumpu koriste upravljački
signali proračunati u prethodnom dijelu vježbe. Modifikovani model je
predstavljen na slici (\ref{slikaMod}).

\begin{center}
    \captionsetup{type=figure}
    \begin{center}
        \includegraphics[width=\textwidth]{slike/slikaMod.png}
        \caption{Modifikovani simulink model}
        \label{slikaMod}
    \end{center}
\end{center}

Sa slike (\ref{posljednja}) možemo vidjeti da je odziv sistema nije u skladu sa očekivanim, što najviše pokazuje zadnja tačka, koja ne odgovara zadatoj vrijednosti. Uzrok ovome je vjerovatno pogrešan model sistema u SIMULINK-u u odnosu na realni model u laboratoriji. Ono što je potrebno napomenuti jeste da je vrijeme simuliranja sistema povecano sa 20s na 110s.

\begin{center}
    \captionsetup{type=figure}
    \begin{center}
        \includegraphics[width=\textwidth]{slike/posljednja.png}
        \caption{Odziv simuliranog sistema sa linearnim kvadratnim regulatorom}
        \label{posljednja}
    \end{center}
\end{center}

\subsection{Ponašanje stvarnog sistema sa dizajniranim LQR}
Kako u sklopu za pripreme za laboratorijsku vježbu nismo odradili ovakvu pripremu koju smo prikazali u izvještaju, to nismo uspijeli dobiti adekvatno ponašanje sistema sa dizajniranim LQR-om. Ideja je da se dobijeni parametri regulatora i ulazi unesu u upravljačku petlju te da se posmatra ponašanje realnog sistema u odnosu na simulirani.

















